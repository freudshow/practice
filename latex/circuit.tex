\documentclass{article}

\usepackage{tikz}%调用宏包tikz
\usepackage{circuitikz}%调用宏包circuitikz

\begin{document}

\begin{figure}[h!]%使用figure环境
  \begin{center}
    \begin{circuitikz}
      \draw (0,0)%坐标(0,0)做为起始点,(0,2)做为终点,绘制电压源。_V_代表电压源,_v=$U_q$_绘制标识。
      to[V,v=$U_q$] (0,2) % 电压源
      to[short] (2,2)%坐标(2,2)做为起始点,(2,0)做为终点,绘制电阻。_R_代表电压源,_R=$R_1$_绘制标识。
      to[R=$R_1$] (2,0) % 电阻
      to[short] (0,0);%注意结尾的分号!
    \end{circuitikz}
    \caption{first circuit.}%添加标题
  \end{center}
\end{figure}

\begin{figure}[h!]
  \begin{center}
    \begin{circuitikz}
      \draw (0,2)%坐标(0,2)做为起始点,(0,0)做为终点,绘制电压源。_V_代表电压源,_v=$U_q$_绘制标识。
      to[V,v=$U_q$] (0,0) %(0,0)为终点。
      to[short] (2,0)%由点(0,0)短接至点(2,0)
      to[R=$R_1$] (2,2) %坐标(2,0)做为起始点,(2,2)做为终点,绘制电阻。_R_代表电压源,_R=$R_1$_绘制标识。
      to[short] (0,2);%由点(2,2)短接至点(0,2).注意结尾的分号!
    \end{circuitikz}
    \caption{second circuit.}
  \end{center}
\end{figure}

\begin{figure}[h!]
  \begin{center}
    \begin{circuitikz}
      \draw (0,0)
      to[V,v=$U_q$] (0,2)   % 电压源
      to[short] (2,2)
      to[R=$R_1$] (2,0)    % 电阻
      to[short] (0,0);
      \draw (2,2)
      to[short] (4,2)
      to[L=$L_1$] (4,0)   % 电感
      to[short] (2,0);
     \end{circuitikz}
    \caption{third circuit.}
  \end{center}
\end{figure}

\begin{figure}[h!]
  \begin{center}
    \begin{circuitikz}
      \draw (0,0)
      to[V,v=$U_q$] (0,2) % 电压源
      to[short] (2,2)
      to[R=$R_1$] (2,0) % 电阻
      to[short] (0,0);
      \draw (2,2)
      to[short] (4,2)
      to[L=$L_1$] (4,0)  % 电感
      to[short] (2,0);
      \draw (4,2)
      to[short] (6,2)
      to[C=$C_1$] (6,0)  % 电容
      to[short] (4,0);
    \end{circuitikz}
    \caption{fourth circuit.}
  \end{center}
\end{figure}

\begin{figure}[h!]
	\begin{center}
		\begin{circuitikz}
			  \draw (0,0) node[npn](npn1) {}
			  (npn1.base) node[anchor=east] {B}
			  (npn1.collector) node[anchor=south] {C}
			  (npn1.emitter) node[anchor=north] {E};
		\end{circuitikz}
 		\caption{fifth circuit.}
  	\end{center}
\end{figure}

\begin{figure}[h!]
  \begin{center}
    \begin{circuitikz}
      \draw (0,0)
      to[V,v=$U_q$,i<^=$i_q$] (0,2) % 电压源
      to[short] (2,2)
      to[R=$R_1$,i^<=$i_R$] (2,0) % 电阻
      to[short,*-] (0,0);
      \draw (2,2)
      to[short,*-] (4,2)
      to[L=$L_1$,i^<=$i_L$] (4,0)  % 电感
      to[short,*-]  (2,0);
      \draw (4,2)
      to[short,*-] (6,2)
      to[C=$C_1$,i^<=$i_C$] (6,0)  % 电容
      to[short] (4,0);
      \draw (4,0) to[short] node[ground] {GND} (4,-0.5);
    \end{circuitikz}
    \caption{sixth circuit.}
  \end{center}
\end{figure}

\end{document}